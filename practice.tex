\documentclass[submit,techrep,noauthor]{ipsj}
\usepackage[dvipdfmx]{graphicx}
\usepackage{url}
\usepackage{amsmath}
%\usepackage{newtxtext, newtxmath}
\def\Underline{\setbox0\hbox\bgroup\let\\\endUnderline}
\def\endUnderline{\vphantom{y}\egroup\smash{\underline{\box0}}\\}
\def\|{\verb|}

\setcounter{巻数}{61} %vol53=2012
\setcounter{号数}{10}
\setcounter{page}{1}
 
\begin{document}
\title{\LaTeX 練習課題文書}
\affiliate{SU}{静岡大学\\
Shizuoka University}
\author{電波 届}{Denpa Todoku}{SU}

\begin{abstract}
  この文書は初めて\LaTeX に触れる学生がレポートらしいものを作成するためのサンプルです.
\end{abstract}

\maketitle

\section{はじめに}

このサンプル文書は,情報処理学会の研究会用のスタイルに基づいて書かれています.
スタイルファイル(クラスファイル)は,情報処理学会論文誌のWebページより入手できます.
スタイルファイルとともに配布されているtech-jsample.texという名前のスタイルを参考に作れば良いでしょう.

% 続きは自分で書くこと。


% 参考文献リスト

\begin{thebibliography}{99}
  \bibitem{conf-example} Shrivastava, V., Nabeel, A., Rayanchu, S.,
    Banerjee, S., Keshav, S., Papagiannaki, K., and Mishra A.: CENTAUR:
    Realizing the Full Potential of Centralized WLANs through a Hybrid Data
    Path, \textit{Proc.\ the 15th Annual International Conference on Mobile Computing
    and Networking (MobiCom '09)}, pp.~297--308 (2009).

  % 続きは自分で書くこと

\end{thebibliography}

\end{document}
